\documentclass{report}

\usepackage{amsmath}
\usepackage{amssymb}
\usepackage{mathtools}
\usepackage{tcolorbox}
\usepackage{enumerate}
\usepackage{titlesec}

\def\C{\mathbb{C}}
\def\R{\mathbb{R}}
\def\Z{\mathbb{Z}}

\DeclarePairedDelimiter\abs{\lvert}{\rvert}%
\renewcommand{\exp}[1]{\operatorname{exp}\left(#1\right)}
\newcommand{\Log}[1]{\operatorname{Log}\left(#1\right)}
\newcommand{\Arg}[1]{\operatorname{Arg}\left(#1\right)}
\titleformat*{\section}{\normalsize\bfseries}

\title{Week 5 summary}
\author{Daniel Gallo, Esther Jerez, Ole Kristian}

\begin{document}
    \maketitle
    \noindent
    Assume we have a function $f(z)$, written in the form:
    \begin{align*}
        f(z) = u(x,y) + iv(x,y)
    \end{align*}
    and $f'(x)$ exists. Then we can write the derivative as:
    \begin{align*}
        f'(z) = u_x(x,y) + iv_x(x,y) = -iu_y(x,y) + v_y(x,y)
    \end{align*}
    Notice that we have applied the Cauchy-Riemann equations ($u_x = v_y$ and $u_y = -v_x$) in the last step of the equality.
    
    \section*{Cauchy-Riemann equations in polar form}
    Assume we have:
    \begin{align*}
        x = r\cos(\theta) \\
        y = r\sin(\theta)
    \end{align*}
    By applying the chain rule we get:
    \begin{align*}
        \dfrac{du}{dr} &= u_r \\ &= u_x\dfrac{dx}{dr} + u_y\dfrac{dy}{dr} \\ &= u_x\cos(\theta)+u_y\sin(\theta) \\
        \dfrac{du}{d\theta} &= u_\theta \\ &= r(u_x(-\sin(\theta)) + u_y\cos(\theta))
    \end{align*}
    Solving for $u_x$ and $u_y$ we get:
    \begin{align*}
        u_x = u_r\cos(\theta) - u_\theta\dfrac{1}{r}\sin(\theta) \\
        u_y = u_r\sin(\theta) + u_\theta\dfrac{1}{r}\cos(\theta)
    \end{align*}
    Applying this with the Cauchy-Riemann equations to the previous expressions of $u_r$ and $u_\theta$ we obtain:
    \begin{align*}
        ru_r &= v_\theta \\ u_\theta &= -rv_r
    \end{align*}
    \begin{tcolorbox}[title=Theorem]
        For a function $f(z)$ to be differentiable in a point $z_0$ it needs to fulfill 3 conditions:
        \begin{enumerate}
            \item $u_x, u_y, v_x, v_y$ all need to exist in some neighbourhood of $z_0$.
            \item $u_x, u_y, v_x, v_y$ all need to be continuous at $z_0$.
            \item $u_x, u_y, v_x, v_y$ all need to satisfy the Cauchy-riemann equations at $z_0$.
        \end{enumerate}
    \end{tcolorbox}
    \section*{Example 1}
    We want to know if $f(z) = \exp{z}$ is differentiable. \\
    We can write the function as folliwing:
    \begin{align*}
        f(z) = \exp{z} = \exp{x+iy} = \exp{x}\exp{iy} = \exp{x}(\cos(y) + i\sin(y))
    \end{align*}
    Then we obtain that:
    \begin{align*}
        u &= \exp{x}\cos(y) \\
        v &= \exp{x}\sin(y)
    \end{align*}
    \noindent
    These equations are continuous in all partial point because we know that $\exp{x}, \cos(y)$ and $\sin(y)$ are continuous for everyx, which satisfies contitions (1) and (2) of the previous theorem.
    \\ Let us check the third condition:
    \begin{align*}
        u_x &= \exp{x}\cos(y) = v_y = \exp{x}\cos(y) \\
        u_y &= -\exp{x}\sin(y) = -v_x = -(\exp{x}\sin(y))
    \end{align*}
    So we can conclude that $f(z)$ is differentiable.\\
    
    \section*{Analytic functions}
    We say that a function $f$ is analytic(or holomorphic):
    \begin{itemize}
        \item On an open set $S$ if $f$ is differentiable on $S$.
        \item At $z_0$ if it is analytic in some $\epsilon$-neighborhood of $z_0$.
        \item If a function is analytic in $\C$, then we say $f$ is entire.
    \end{itemize}

    \begin{tcolorbox}[title=Theorem]
        Assume we have a function $f(z) = u(x,y) + iv(x,y)$. If f is analytic in a domain $D$, then both $u$ and $v$ are harmonic.\\
        Remember: $u\colon\R^2 \longrightarrow \R$ is called a harmonic function if $\Delta u = u_{xx} + u_{yy} = 0$.
    \end{tcolorbox}
    \section*{Example 2}
    We want to know if $u(x,y) = 3x^2y-y^3$ is the real part of an analytic function.\\
    For the previous theorem we know that if $u$ is not harmonic, then it can not be the real part of an analytic function. But in case it is harmonic, it doesn't mean it is the real part of an analytic functions. So in this type of exercise we have two steps:
    \begin{enumerate}
        \item Check if $u$ is harmonic.
        \item If not, then it is not the real part of an analytic function. If yes, then we have to find the imaginary part of the analytic function we are looking for. This imaginary part must solve Cauchy-Riemann equations.
    \end{enumerate}
    Let us start checking if it is harmonic:
    \begin{align*}
        u_x &= 6xy \quad \quad u_y = 3x^2-3y^2 \\
        u_{xx} &= 6y \quad \quad u_{yy} = -6y \\
    \end{align*}
    As we can check, $\Delta u = 0 $ so  $u$ is harmonic. Now we try to find the imaginary part $v(x,y)$ solving the C.R. equations:
    \begin{align*}
        u_x &= 6xy = v_y \\
        u_y &= 3x^2-3y^2 = -v_x
    \end{align*}
    Taking the first equation and integrating:\\
    \begin{equation*}
        v(x,y)=\int_{0}^y6xydt+h(x)=3xy^2+h(x)
    \end{equation*}
    Now differentiating it in function of $x$ and applying the second equation:\\
    \begin{equation*}
        v_x= 3y^2 + h'(x) = -u_y = -3x^2+3y^2
    \end{equation*}
    Taking everything into consideration we obtain that $v(x,y) = 3xy^2 - x^3 + C$ ($C$ is a constant) and $f(x,y)= 3x^2y-y^3+i(3xy^2 - x^3 + C)$. So yes, $u(x,y)$ is the real part of an analytic function.
    \section*{Chapter 3: logarithm on complex numbers}
    In the complex plane we define the logarithm as a multi-valued function:
    \begin{equation*}
        \log{z} = \ln{\abs{z}} + i\arg{z}
    \end{equation*}
     Assume $z = r\exp{i\theta}$. Then we can write the logarithm as follows:
     \begin{equation*}
        \log{z} = \ln{r} + i2\pi k, k\in \Z
     \end{equation*}
     We then define \textbf{the principal logarithm} as:
     \begin{equation*}
        \Log{z} = \ln{r} + i\Arg{z}
     \end{equation*}
     and it is defined for $z\neg 0$.
     \section*{Example 3}
     Here we will see some examples of the new definition of logarithm in the complex numbers:
     \begin{itemize}
         \item $\log{1} = \ln{1} + i\arg{1} = i2\pi k$
         \item $\log{-1} = \ln{\abs{1}} + i\arg{-1} = i\pi i2\pi k$
         \item $\log{1+i} = \ln{\abs{1+i}} + i\arg{1+i} = \ln{\sqrt{2}} +i\dfrac{\pi}{4} + i2\pi k$
     \end{itemize}
\end{document}