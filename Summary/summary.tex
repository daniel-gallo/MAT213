\documentclass{article}

\usepackage{amsmath}
\usepackage{amssymb}
\usepackage{mathtools}
\usepackage{enumerate}
\usepackage[theorems]{tcolorbox}
\usepackage{tcolorbox}
\usepackage{hyperref}

\newtcbtheorem[number within=section]%
{theorem} % \begin..
{Theorem} % Title
{} % Style - default
{thm} % label prefix; cite as ``theo:yourlabel''
\hypersetup{colorlinks=true,linkcolor=blue}

\def\C{\mathbb{C}}
\def\R{\mathbb{R}}
\def\Z{\mathbb{Z}}

\DeclarePairedDelimiter\abs{\lvert}{\rvert}%
\renewcommand{\exp}[1]{\operatorname{exp}\left(#1\right)}
\newcommand{\Log}[1]{\operatorname{Log}\left(#1\right)}
\newcommand{\Arg}[1]{\operatorname{Arg}\left(#1\right)}

\title{Week 5 Summary}
\author{
    Bombelli \\
    \textit{Daniel Gallo, Esther Jerez, Ole Kristian}
}

\begin{document}
    \maketitle
    \noindent
    
    \section{Cauchy-Riemann equations}
    Assume we have a function $f(z)$, written in the form:
    \begin{align*}
        f(z) = u(x,y) + iv(x,y)
    \end{align*}
    and $f'(z)$ exists. Then, we can write $f'(z)$ as
    \begin{equation*}
        f'(z) = u_x(x,y) + iv_x(x,y) = -iu_y(x,y) + v_y(x,y)
    \end{equation*}
    
    \section{Cauchy-Riemann equations in polar form}
    Assume we have:
    \begin{align*}
        x = r\cos(\theta) \\
        y = r\sin(\theta)
    \end{align*}
    Applying the chain rule we get:
    \begin{align*}
        \frac{du}{dr} &= \frac{du}{dx}\frac{dx}{dr} + \frac{du}{dy}\frac{dy}{dr} \\ 
        &= u_x\cos(\theta)+u_y\sin(\theta) \\
        \frac{du}{d\theta} &= \frac{du}{dx}\frac{dx}{d\theta} + \frac{du}{dy}\frac{dy}{d\theta} \\ 
        &= -ru_x\sin(\theta)+ru_y\cos(\theta)
    \end{align*}
    We can do the same thing with $v$. Applying the Cauchy-Riemann equations to these expressions we obtain:
    \begin{align*}
        ru_r &= v_\theta \\ u_\theta &= -rv_r
    \end{align*}

    \section{Sufficient conditions for differentiability}
    \begin{theorem}{Sufficient conditions for differentiability}{sufficient_conditions}
        If the three conditions below hold, then $f(z)$ is differentiable at $z_0$. 
        \begin{enumerate}
            \item $u_x, u_y, v_x, v_y$ exist in some neighbourhood of $z_0$.
            \item $u_x, u_y, v_x, v_y$ are continuous at $z_0$.
            \item $u_x, u_y, v_x, v_y$ satisfy the Cauchy-Riemann equations at $z_0$.
        \end{enumerate}
    \end{theorem}
    \subsection*{Example}
    We want to know if $f(z) = \exp{z}$ is differentiable. We can write the $f$ as:
    \begin{equation*}
        f(z) = \exp{z} = \exp{x+iy} = \exp{x}\exp{iy} = \exp{x}(\cos(y) + i\sin(y))
    \end{equation*}
    Then we obtain that:
    \begin{equation*}
        u = \exp{x}\cos(y)
    \end{equation*}
    \begin{equation*}
        v = \exp{x}\sin(y)
    \end{equation*}
    \noindent
    Both $u$ and $v$ are $C^\infty$, so we just have to check the Cauchy-Riemann equations to apply \ref{thm:sufficient_conditions}.
    \begin{align*}
        u_x &= \exp{x}\cos(y) = v_y = \exp{x}\cos(y) \\
        u_y &= -\exp{x}\sin(y) = -v_x = -\exp{x}\sin(y)
    \end{align*}
    Since all three conditions hold, we can conclude that $f(z)$ is differentiable.

    \section{Analytic functions}
    We say that a function $f$ is analytic (or holomorphic) at $z_0$ if it has a derivative at each point in some neighborhood of $z_0$. If it is analytic at every point of an open set $S$, we say that $f$ in analytic in $S$.

    \section{Harmonic functions}
    We say that $u\colon\R^2 \longrightarrow \R$ is harmonic if $\Delta u = u_{xx} + u_{yy} = 0$.

    \begin{theorem}{Components of an analytic function are harmonic}{components_are_harmonic}
        Assume we have a function $f(z) = u(x,y) + iv(x,y)$. If f is analytic in a domain $D$, then both $u$ and $v$ are harmonic.
    \end{theorem}
    \subsection*{Example}
    We want to know if $u(x,y) = 3x^2y-y^3$ is the real part of an analytic function. From the previous theorem we know that if $u$ is not harmonic, then it cannot be the real part of an analytic function. But if $u$ is harmonic, it doesn't necessarily mean that there exists such $f$. We will look for $f$ explicitly by integrating. \\
    Let us start checking if it is harmonic:
    \begin{align*}
        u_x &= 6xy \quad \quad u_y = 3x^2-3y^2 \\
        u_{xx} &= 6y \quad \quad u_{yy} = -6y \\
    \end{align*}
    As we can check, $\Delta u = 0 $ so $u$ is harmonic. Now we try to find the imaginary part $v(x,y)$ solving the Cauchy-Riemann equations:
    \begin{align*}
        u_x &= 6xy = v_y \\
        u_y &= 3x^2-3y^2 = -v_x
    \end{align*}
    Taking the first equation and integrating
    \begin{equation*}
        v(x,y)=\int6xydy=3xy^2+h(x)
    \end{equation*}
    Now differentiating it with respect to $x$ and applying using the second equation:\\
    \begin{equation*}
        v_x= 3y^2 + h'(x) = -u_y = -3x^2+3y^2
    \end{equation*}
    Taking everything into consideration we obtain that $v(x,y) = 3xy^2 - x^3 + C$ ($C$ is a constant) and $f(x,y)= 3x^2y-y^3+i(3xy^2 - x^3 + C)$. So yes, $u(x,y)$ is the real part of an analytic function.
    \section{Complex logarithm}
    In the complex plane we define the logarithm as a multi-valued function.
    \begin{align*}
        \log\colon \C \setminus \{0\} &\to \C \\
        z &\mapsto \log{\abs{z}} + i\arg{z}
    \end{align*}
    If $z = r\exp{i\theta}$, we can write the logarithm has follows:
    \begin{equation*}
       \log{z} = \log{r} + i(\theta + 2k\pi), \quad k \in \Z
    \end{equation*}
    The \textbf{principal logarithm}, $\Log{z}$, has imaginary part in $(-\pi, \pi]$
\end{document}