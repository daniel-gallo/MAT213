\documentclass{article}

\usepackage{tcolorbox}
\usepackage{amsmath}
\usepackage{amssymb}
\usepackage{enumitem}

\def\C{\mathbb{C}}
\def\R{\mathbb{R}}
\renewcommand{\exp}[1]{\operatorname{exp}\left(#1\right)}
\setlength{\parindent}{0em}

\title{Mandatory assignment 2}
\author{Daniel Gallo}

\begin{document}
    \maketitle

    \begin{tcolorbox}[title=Exercise 1]
        Find $z$ in Cartesian and polar coordinates.
        \begin{equation*}
            (-3 + 4i) z = -14 + 2i
        \end{equation*}
    \end{tcolorbox}
    \begin{equation*}
        z = \frac{-14 + 2i}{-3 + 4i} = \frac{-14 + 2i}{-3 + 4i} \frac{-3 -4i}{-3 - 4i} = \frac{50 + 50i}{25} = 2 + 2i = 2\sqrt{2} \exp{i\frac{\pi}{4}}
    \end{equation*}
    
    \begin{tcolorbox}[title=Exercise 2]
        Consider the function $f \colon \C \to \C$ defined by
        \begin{equation*}
            f(z) = e^z + z^2
        \end{equation*}
        \begin{enumerate}[label=(\alph*)]
            \item Is the function multivalued?
            \item Write $f(z)$ as $u(x, y) + iv(x, y)$. Show that $u$ and $v$ satisfy the Cauchy-Riemann equations.
        \end{enumerate}    
    \end{tcolorbox}
    The function $f$ \textbf{is not multivalued} because neither $\exp{z}$ nor $z^2$ are multivalued functions.
    \begin{align*}
        f(z) &= e^z + z^2 = e^x e^y + (x + iy)^2 \\
        &= e^x (\cos{y} + i\sin{y}) + x^2 - y^2 + 2xyi \\
        &= \underbrace{x^2 - y^2 + e^x\cos{y}}_{u(x, y)} + i\underbrace{(2xy + e^x\sin{y})}_{v(x, y)}
    \end{align*}
    The Cauchy-Riemann equations hold because $f$ is \textbf{analytic}. Nonetheless, we can check explicitly that $u_x = v_y$ and $u_y = -v_x$.
    \begin{align*}
        &u_x = 2x + e^x\cos{y}  & v_x = 2y + e^x\sin{y} \\
        &u_y = -2y - e^x\sin{y} & v_y = 2x + e^x\cos{y}
    \end{align*}

    \begin{tcolorbox}[title=Exercise 3]
        Let $D$ be a domain in $\C$.
        \begin{enumerate}[label=(\alph*)]
            \item Let $f \colon D \to \C$ be analytic. Assume the values of $f$ are either real or purely imaginary. Show that $f$ has to be constant.
            \item Let $u \colon D \to \R$ be a harmonic function. Show that any harmonic conjugate of $u$ will also be harmonic. Show that the harmonic conjugate is unique up to a constant.
            \item Let $f \colon D \to \C$ be an analytic function. For any $z_0 \in D$ define $w_0 = f(z_0) = u_0 + iv_0$. Find the angle the lines $u(x, y) = u_0$ and $v(x, y) = v_0$ make at $w_0$.
        \end{enumerate}
    \end{tcolorbox}
    \begin{enumerate}[label=(\alph*)]
        \item If $f$ is purely real, $v$ will be identically zero. Applying the Cauchy-Riemann equations (which hold, since $f$ is analytic in $D$), we get that $u_x = u_y = 0$.  If $f$ is purely imaginary, $u$ will be identically zero, and applying the Cauchy-Riemann equations, we get that $v_x = v_y = 0$. In either case we can conclude that $u$ ($v$) is a constant.
        \item To prove that $v$ is harmonic we have to show that $v_{xx} + v_{yy} = 0$. If we differentiate $u_x = v_y$ with respect to $y$ we get
        \begin{equation}
            \label{eq:vyy}
            u_{xy} = v_{yy}
        \end{equation}
        Differentiating $u_y = -v_x$ with respect to $x$ we get
        \begin{equation}
            \label{eq:vxx}
            u_{yx} = -v_{xx}
        \end{equation}
        Now we can clearly see from \eqref{eq:vyy} and \eqref{eq:vxx} that $v_{xx} + v_{yy} = -u_{yx} + u_{xy}$. Using Schwartz Theorem, we conclude that $v$ is indeed harmonic. \par
        Assume that $v$ and $\bar{v}$ are harmonic conjugates of $u$. That means that $f = u + iv$ and $\bar{f} = u + i\bar{v}$ are analytic functions in $D$.
        \begin{equation*}
            \bar{f} - f = i(\bar{v} - v)
        \end{equation*}
        The function $\bar{f} - f$ is analytic because it is the difference of two analytic functions and purely imaginary, which means that $\bar{v} = v + C$.
        \item TODO: ask about this
    \end{enumerate}
\end{document}