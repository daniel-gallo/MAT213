\documentclass{article}

\usepackage{tcolorbox}
\usepackage{amsmath}
\usepackage{amssymb}
\usepackage{enumitem}

\def\C{\mathbb{C}}
\def\R{\mathbb{R}}
\def\N{\mathbb{N}}
\renewcommand{\exp}[1]{\operatorname{exp}\left(#1\right)}
\setlength{\parindent}{0em}

\title{Mandatory assignment 2}
\author{Daniel Gallo}

\begin{document}
    \maketitle

    \begin{tcolorbox}[title=Exercise 1]
        Find $z$ in Cartesian and polar coordinates.
        \begin{equation*}
            (-3 + 4i) z = -14 + 2i
        \end{equation*}
    \end{tcolorbox}
    \begin{equation*}
        z = \frac{-14 + 2i}{-3 + 4i} = \frac{-14 + 2i}{-3 + 4i} \frac{-3 -4i}{-3 - 4i} = \frac{50 + 50i}{25} = 2 + 2i = 2\sqrt{2} \exp{i\frac{\pi}{4}}
    \end{equation*}
    
    \begin{tcolorbox}[title=Exercise 2]
        Consider the function $f \colon \C \to \C$ defined by
        \begin{equation*}
            f(z) = e^z + z^2
        \end{equation*}
        \begin{enumerate}[label=(\alph*)]
            \item Is the function multivalued?
            \item Write $f(z)$ as $u(x, y) + iv(x, y)$. Show that $u$ and $v$ satisfy the Cauchy-Riemann equations.
        \end{enumerate}    
    \end{tcolorbox}
    The function $f$ \textbf{is not multivalued} because neither $\exp{z}$ nor $z^2$ are multivalued functions.
    \begin{align*}
        f(z) &= e^z + z^2 = e^x e^y + (x + iy)^2 \\
        &= e^x (\cos{y} + i\sin{y}) + x^2 - y^2 + 2xyi \\
        &= \underbrace{x^2 - y^2 + e^x\cos{y}}_{u(x, y)} + i\underbrace{(2xy + e^x\sin{y})}_{v(x, y)}
    \end{align*}
    The Cauchy-Riemann equations hold because $f$ is \textbf{analytic}. Nonetheless, we can check explicitly that $u_x = v_y$ and $u_y = -v_x$.
    \begin{align*}
        &u_x = 2x + e^x\cos{y}  & v_x = 2y + e^x\sin{y} \\
        &u_y = -2y - e^x\sin{y} & v_y = 2x + e^x\cos{y}
    \end{align*}

    \begin{tcolorbox}[title=Exercise 3]
        Let $D$ be a domain in $\C$.
        \begin{enumerate}[label=(\alph*)]
            \item Let $f \colon D \to \C$ be analytic. Assume the values of $f$ are either real or purely imaginary. Show that $f$ has to be constant.
            \item Let $u \colon D \to \R$ be a harmonic function. Show that any harmonic conjugate of $u$ will also be harmonic. Show that the harmonic conjugate is unique up to a constant.
            \item Let $f \colon D \to \C$ be an analytic function. For any $z_0 \in D$, we have $f(z_0) = f(x_0 + iy_0) = u_0 + iv_0$. Find the angle the lines $u(x, y) = u_0$ and $v(x, y) = v_0$ make at $z_0$.
        \end{enumerate}
    \end{tcolorbox}
    \begin{enumerate}[label=(\alph*)]
        \item If $f$ is purely real, $v$ will be identically zero. Applying the Cauchy-Riemann equations (which hold, since $f$ is analytic in $D$), we get that $u_x = u_y = 0$.  If $f$ is purely imaginary, $u$ will be identically zero, and applying the Cauchy-Riemann equations, we get that $v_x = v_y = 0$. In either case we can conclude that $u$ ($v$) is a constant.
        \item To prove that $v$ is harmonic we have to show that $v_{xx} + v_{yy} = 0$. If we differentiate $u_x = v_y$ with respect to $y$ we get
        \begin{equation}
            \label{eq:vyy}
            u_{xy} = v_{yy}
        \end{equation}
        Differentiating $u_y = -v_x$ with respect to $x$ we get
        \begin{equation}
            \label{eq:vxx}
            u_{yx} = -v_{xx}
        \end{equation}
        Now we can clearly see from \eqref{eq:vyy} and \eqref{eq:vxx} that $v_{xx} + v_{yy} = -u_{yx} + u_{xy}$. Using Schwartz Theorem, we conclude that $v$ is indeed harmonic. \par
        Assume that $v$ and $\bar{v}$ are harmonic conjugates of $u$. That means that $f = u + iv$ and $\bar{f} = u + i\bar{v}$ are analytic functions in $D$.
        \begin{equation*}
            \bar{f} - f = i(\bar{v} - v)
        \end{equation*}
        The function $\bar{f} - f$ is analytic because it is the difference of two analytic functions and purely imaginary, which means that $\bar{v} = v + C$.
        \item Recall that if we have a curve $u(x, y) = u_0$, the normal vector at $(x_0, y_0)$ is given by the gradient at that point $\nabla u (x_0, y_0)$. Applying the Cauchy-Riemann equations we can easily see that the normal vectors of $u$ and $v$ are perpendicular at $z_0$.
        \begin{equation*}
            \nabla u \cdot \nabla v = (u_x, u_y) \cdot (v_x, v_y) = u_x v_x + u_y v_y = 0
        \end{equation*}
        The angle between two curves at a given point is defined as the angle between the tangent vectors, but since we know that the normal vectors are perpendicular, we can conclude that the tangent vectors are perpendicular as well. In fact, the family of curves $\{u(x, y) = a\}_{a \in \R}$ is orthogonal to the family $\{v(x, y) = b\}_{b \in \R}$, because we can use the same reasoning for any intersection point of a curve of the first family with a curve of the second.
    \end{enumerate}

    \begin{tcolorbox}[title=Exercise 4]
        Compute the values. If multiple valued, give all values.
        \begin{enumerate}[label=(\alph*)]
            \item $\exp{\ln{2} + i\pi}$
            \item $\cos^{-1}{i}$
            \item $i^{2i}$
        \end{enumerate}
    \end{tcolorbox}
    \begin{enumerate}[label=(\alph*)]
        \item \begin{equation*}
            \exp{\ln{2} + i\pi} = \exp{\ln{2}}\exp{i\pi} = -2
        \end{equation*}
        \item Recall that
        \begin{equation*}
            \cos{z} = \frac{\exp{iz} + \exp{-iz}}{2}
        \end{equation*}
        Making that equal to $i$ and using the substitution $w = \exp{iz}$ we get
        \begin{equation*}
            w + w^{-1} = 2i
        \end{equation*}
        Multiplying everything by $w$ and rearranging the terms we get to the following quadratic:
        \begin{equation*}
            w^2 - 2iw + 1 = 0
        \end{equation*}
        Solving it we get that $\exp{iz} = w = i(1 \pm \sqrt{2})$. Isolating the $z$ we get our answer ($k \in \N$).
        \begin{align*}
            z &= \left(\frac{\pi}{2} + 2k\pi\right) -i\log(\sqrt{2} + 1) \\
            z &= \left(\frac{3\pi}{2} + 2k\pi\right) -i\log(\sqrt{2} - 1) \\
        \end{align*}
        \item \begin{equation*}
            i^{2i} = \exp{i \frac{\pi}{2}}^{2i} = \exp{-\pi}
        \end{equation*}
    \end{enumerate}

    \begin{tcolorbox}[title=Exercise 5]
        Compute $\int_C f(z) dz$
        \begin{equation*}
            f(z) = \sin(ix) + y\cos(y)
        \end{equation*}
    \end{tcolorbox}
    Note that we can rewrite $f$ as
    \begin{equation*}
        f(z) = y\cos(y) + i\sinh(x)
    \end{equation*}
    Using the Cauchy-Riemann equations we can easily see that $F$ is an antiderivative of $f$.
    \begin{equation*}
        F(z) = xy\cos(y) + i\cosh(x)
    \end{equation*}
    Thus, $\int_C f(z) dz = F(\pi/2 + i\pi/2) - F(0) = i\cosh(\pi/2) - i$.

    \begin{tcolorbox}[title=Exercise 6]
        Let $C$ be the semicircle with parametrization $z(\theta) = 2\exp{i\theta}$, with $\theta \in [\frac{\pi}{2}, \frac{3\pi}{2}]$.
        \begin{enumerate}[label=(\alph*)]
            \item Show that
            \begin{equation*}
                \int_C{\exp{\frac{1}{z}}}dz \leq 2\pi
            \end{equation*}
            \item Show that
            \begin{equation*}
                \int_{\frac{\pi}{2}}^{\frac{3\pi}{2}} e^{2\cos{\theta}}(\cos(2\sin{\theta})\cos{\theta} - \sin(2\sin{\theta})\sin{\theta}) d\theta = -\sin{2}
            \end{equation*}
        \end{enumerate}
    \end{tcolorbox}
    \begin{enumerate}[label=(\alph*)]
        \item Note that the length of $C$ is $2\left(\frac{3\pi}{2} - \frac{\pi}{2}\right) = 2\pi$. Then, note that:
        \begin{equation*}
            \left|\frac{1}{z}\right| = \Re\left(\frac{1}{z}\right) = \frac{x}{x^2 + y^2} < 1
        \end{equation*}
        Thus, using the Theorem we saw in class, 
        \begin{equation*}
            \int_C{\exp{\frac{1}{z}}}dz \leq 2\pi
        \end{equation*}
        \item On the one hand,
        \begin{equation*}
            \int_C e^z dz = e^z \Bigr]^{2i}_{-2i} = e^{2i} - e^{-2i} = -2i\sin{2}
        \end{equation*}
        On the other hand,
        \begin{align*}
            \int_C e^z dz &= \int_{\frac{\pi}{2}}^{\frac{3\pi}{2}} e^{2\exp{i\theta}} 2 i e^{i \theta} \\
            &= 2i\int_{\frac{\pi}{2}}^{\frac{3\pi}{2}} e^{2\cos{\theta}} \left(\cos(\theta + 2\sin \theta) + i\sin(\theta + 2\sin \theta)\right) d\theta \\
            &= 2i\int_{\frac{\pi}{2}}^{\frac{3\pi}{2}} e^{2\cos{\theta}} \cos(\theta + 2\sin \theta) d\theta\\
            &= 2i\int_{\frac{\pi}{2}}^{\frac{3\pi}{2}} e^{2\cos{\theta}} \cos(\theta + 2\sin \theta) \\
            &= 2i \int_{\frac{\pi}{2}}^{\frac{3\pi}{2}} e^{2\cos{\theta}}(\cos(2\sin{\theta})\cos{\theta} - \sin(2\sin{\theta})\sin{\theta}) d\theta \\
        \end{align*}
        In the third step we have ignored $i\sin(\theta + 2\sin \theta))$ because the result is purely imaginary. In the last one we have used the formula for $\cos(\alpha + \beta)$. Using both results, we can conclude that
        \begin{equation*}
            \int_{\frac{\pi}{2}}^{\frac{3\pi}{2}} e^{2\cos{\theta}}(\cos(2\sin{\theta})\cos{\theta} - \sin(2\sin{\theta})\sin{\theta}) d\theta = -\sin{2}
        \end{equation*}
    \end{enumerate}

    \begin{tcolorbox}[title=Exercise 7]
        Let $C$ be the circle with parametrization $z(\theta) = -1 + \exp{i\theta}$. Compute $\int_C{f(z)dz}$ for the following functions:
        \begin{enumerate}[label=(\alph*)]
            \item \begin{equation*}
                f(z) = \frac{1}{1 - z^2}
            \end{equation*}
            \item \begin{equation*}
                f(z) = \frac{1}{1 + z^2}
            \end{equation*}
            \item \begin{equation*}
                f(z) = \frac{\cos{z}}{2 + z - z^2}
            \end{equation*}
        \end{enumerate}
    \end{tcolorbox}
    
    \begin{enumerate}[label=(\alph*)]
        \item Let $C_\rho = -1 + \rho\exp{i\theta}$, with $p \in (0, 1)$. Because of the corollary seen in class, we have
        \begin{align*}
            \int_C \frac{1}{1 - z^2}dz &= \int_{C_\rho} \frac{1}{1 - z^2}dz \\
            &= \int_{C_\rho} \frac{1}{(1 - z)(1 + z)}dz \\
            &= \int_0^{2\pi} \frac{i\rho\exp{i\theta}}{(2 - \rho\exp{i\theta})\rho\exp{i\theta}}d\theta \\
            &\overset{\rho \to 0}{=} \int_0^{2\pi} \frac{i}{2}dz = \pi i \\
        \end{align*}
        \item Since $f$ only has singularities in $\pm i$, which is outside the region enclosed by $C$, we can conclude by Cauchy–Goursat that
        \begin{equation*}
            \int_C \frac{1}{1 + z^2}dz = 0
        \end{equation*}
        \item Let $C_\rho = -1 + \rho\exp{i\theta}$, with $p \in (0, 1)$. Because of the corollary seen in class, we have
        \begin{align*}
            \int_C \frac{1}{2 + z - z^2}dz &= \int_{C_\rho} \frac{1}{2 + z - z^2}dz \\
            &= \int_{C_\rho} \frac{\cos{z}}{(2 - z)(1 + z)}dz \\
            &= \int_0^{2\pi} \frac{i\rho\exp{i\theta}\cos(-1 + \rho\exp{i\theta})}{(3 - \rho\exp{i\theta})\rho\exp{i\theta}}d\theta \\
            &\overset{\rho \to 0}{=} \int_0^{2\pi} \frac{i\cos(-1)}{3}dz = \frac{2\pi\cos(-1)i}{3} \\
        \end{align*}
    \end{enumerate}
\end{document}