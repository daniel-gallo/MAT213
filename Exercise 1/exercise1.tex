\documentclass{report}

\usepackage{amsmath}
\usepackage{amssymb}
\usepackage{mathtools}
\usepackage{tcolorbox}
\usepackage{enumerate}

\def\C{\mathbb{C}}
\def\R{\mathbb{R}}
\def\N{\mathbb{N}}
\DeclarePairedDelimiter\abs{\lvert}{\rvert}%
\DeclarePairedDelimiter\ceil{\lceil}{\rceil}
\DeclarePairedDelimiter\floor{\lfloor}{\rfloor}
\renewcommand{\exp}[1]{\operatorname{exp}\left(#1\right)}
\renewcommand{\arctan}[1]{\operatorname{arctan} \left(#1\right)}

\title{Exercise 1}
\author{Daniel Gallo, Esther Jerez, Ole Kristian}

\begin{document}
    \maketitle
    
    \section*{Exercise 1}
    \begin{tcolorbox}[title=Part a]
        Find all solutions $z$ of
        \begin{equation*}
            z^3 = 4i - 4
        \end{equation*}
    \end{tcolorbox}
    \noindent
    Using polar form,
    \begin{equation*}
        z^3 = 4i - 4 = 4\sqrt{2}\exp{\frac{3\pi}{4}i}
    \end{equation*}
    We know that the principal root will have radius $r = \sqrt[3]{4\sqrt{2}}$ and argument $\theta_1 = \frac{\pi}{4}$. The other two roots will be evenly spaced throughout the circle of radius $r$. Thus,
    \begin{align*}
        \theta_1 &= \frac{\pi}{4} \\
        \theta_2 &= \theta_1 + \frac{2\pi}{3} = \frac{11\pi}{12} \\
        \theta_3 &= \theta_2 + \frac{2\pi}{3} = \frac{19\pi}{12} = \frac{-5\pi}{12} \mod{2\pi}\\
    \end{align*}
    \begin{tcolorbox}[title=Part b]
        Write the numbers below on rectangular form
        \begin{equation*}
            (\sqrt{3} - i)^{10} \quad (2 - i)^{9}
        \end{equation*}
    \end{tcolorbox}
    \begin{align*}
        (\sqrt{3} - i)^{10} &= \left(2\exp{-\frac{\pi}{6}i}\right)^{10} = 2^{10}\exp{-\frac{5\pi}{3}i} \\
        &= 2^{10}\left(\frac{1}{2} + \frac{\sqrt{3}}{2}i\right) = 512 + 512\sqrt{3}i
    \end{align*}
    \begin{equation*}
        (2 - i)^{9} = \left(\sqrt{5}\exp{\arctan{-\frac{1}{2}}i}\right)^{9} = -718 + 1199i
    \end{equation*}
    \section*{Exercise 2}
    \begin{equation*}
        A = \{z \in \C : \abs{z} < 1\}
    \end{equation*}
    $A$ is bounded and open, since for any $x \in A$, $x \in B(x, 1 - \abs{x}) \subset A$. $A$ is a domain and a region, because it is a nonempty open connected set.
    \begin{equation*}
        B = \{z \in \C : \abs{z} = 1\}
    \end{equation*}
    $B$ is bounded and closed, since for any $x \in B^c$, $x \in B(x, \abs{\abs{x} - 1}) \subset B^c$. $B$ is not a region because its interior is empty.
    \begin{equation*}
        C = \{z \in \C : 1 < \Im(z) < 2\}
    \end{equation*}
    $C$ is not bounded and open, since for all $x \in C$, $x \in B(x, \min{\{\Im{x} - 1, 2 - \Im{x}}\}) \subset C$. Because $C$ is connected, we can conclude that it is a domain and a region.
    \begin{equation*}
        D = \{z \in \C : 0 < \abs{z} \leq 1\}
    \end{equation*}
    $D$ is bounded but it isn't open nor closed. It's not closed because $0$ is a limit point of $D$ but it is not included in the set. It's not open because $\forall r > 0$, $B(1, r) \nsubseteq D$. It is a region, since its interior, $\{z \in \C : 0 < \abs{z} < 1\}$ is a domain and it is contained in its closure, $\{z \in \C : \abs{z} \leq 1\}$.
    \begin{equation*}
        E = \N
    \end{equation*}
    $E$ is closed and not bounded. To prove that $E^c$ is open, take any $x \in E^c$ and consider $B(x, r)$. If $\Re{x} \in \N$, pick $r = \Im{x}$. Otherwise, pick 
    \begin{equation*}
        r = \min{\{\Re{x} - \floor{\Re{x}}, \ceil{\Re{x}} - \Re{x}\}}
    \end{equation*}
    It is not a domain nor a region because it is not connected.
    \begin{equation*}
        F = \R
    \end{equation*}
    $F$ is not open because for every $x \in F$ and for every $r > 0$, $B(x, r) \nsubseteq F$. It is closed because for every $x \in F^c$, $B(x, \Im{x}) \subseteq F^c$. Obviously, it is not bounded. It's not a region because its interior is empty.
    \begin{equation*}
        G = \left\{\exp{\frac{i}{n}} : n \in \N\right\}
    \end{equation*}
    $G$ is not open because $B(\exp{i}, 0.1) \nsubseteq G$. $G$ is not closed because $1$ is a limit point of $G$ but it is not included. It is bounded. It is not a region, since its interior is empty.
    \section*{Exercise 3}
    \begin{tcolorbox}[title=Part a]
        Consider the function $w$ given by
        \begin{equation*}
            w = u + vi = \frac{1}{(iz)^2} = \frac{1}{(-y + ix)^2}
        \end{equation*}
        Find $u(x, y)$ and $v(x, y)$
    \end{tcolorbox}
    \noindent
    \begin{align*}
        w &= \frac{1}{(-y + ix)^2} = \frac{1}{y^2 - x^2 -2xyi} \\
        &= \frac{1}{y^2 - x^2 -2xyi} \frac{y^2 - x^2 + 2xyi}{y^2 - x^2 + 2xyi} \\
        &= \underbrace{\frac{y^2 - x^2}{(x^2 + y^2)^2}}_{u(x, y)} + \underbrace{\frac{2xy}{(x^2 + y^2)^2}}_{v(x, y)}i
    \end{align*}
    \begin{tcolorbox}[title=Part b.i]
        Compute the limit
        \begin{equation*}
            \lim_{z \to -2i}{\frac{z^2 + 4}{z + 2i}}
        \end{equation*}
    \end{tcolorbox}
    \begin{equation*}
        \lim_{z \to -2i}{\frac{z^2 + 4}{z + 2i}} = \lim_{z \to -2i}{\frac{(z + 2i)(z - 2i)}{z + 2i}} = -4i
    \end{equation*}
    \begin{tcolorbox}[title=Part b.ii]
        Compute the limit
        \begin{equation*}
            \lim_{z \to -2i}{\frac{z^2 + 2}{z + 2i}}
        \end{equation*}
    \end{tcolorbox}
    \noindent
    The numerator tends to $-2$ and the denominator to $0$, so
    \begin{equation*}
        \lim_{z \to -2i}{\frac{z^2 + 2}{z + 2i}} = \infty
    \end{equation*}
    \begin{tcolorbox}[title=Part b.iii]
        Compute the limit
        \begin{equation*}
            \lim_{z \to \infty}{\frac{z + 2}{z + 2i}}
        \end{equation*}
    \end{tcolorbox}
    \begin{equation*}
        \lim_{z \to \infty}{\frac{z + 2}{z + 2i}} = \lim_{z \to \infty}{\frac{z + 2i - 2i + 2}{z + 2i}} = 1 + \lim_{z \to \infty}{\frac{-2i + 2}{z + 2i}} = 1
    \end{equation*}
    The last equality can be explained noticing that the radius of the numerator is bounded and the radius of the denominator is not.
    
    \section*{Exercise 4}
    \begin{tcolorbox}[title=Part a]
        What does it mean that a function $f(z)=u(x,y)+iv(x,y)$ is analytic? Show that if $f$ is analytic, then $u$ and $v$ satisfy the Cauchy-Riemann equations.
    \end{tcolorbox}
    \noindent
    We say that $f$ is analytic:
    \begin{itemize}
        \item On an open set $S$ if $f$ is differentiable on $S$.
        \item At $z_0$ if it is analytic in some $\epsilon$-neighborhood of $z_0$.
    \end{itemize}
    With just the definition we can see that for every point in which $f$ is analytic, it is differentiable. We can now apply a theorem seen in class that affirm that if $f'$ exists, then the Cauchy-Riemann equations hold (for the points in which $f$ was analytic).
    \begin{tcolorbox}[title=Part b]
        What does it mean for a function $u(x,y)$ to be harmonic. Determine if the following functions are harmonic
        \begin{align*}
            (i) \ u(x,y)=x^2+2y^2, \quad (ii) \ u(x,y)=\sin(x)\cosh(y).
        \end{align*}
        Does the above imply/not imply that $u(x,y)$ is the real part of an analytic function?
    \end{tcolorbox}
    \noindent
    $u\colon\R^2 \longrightarrow \R$ is called a harmonic function if $\Delta u = u_{xx} + u_{yy} = 0$.
    \begin{enumerate}[(i)]
        \item $u(x,y)=x^2+2y^2$
        \begin{align*}
            u_x&=2x & u_y&=4y \\
            u_{xx}&=2 & u_{yy}&=4 \\
        \end{align*}

        $\Delta u = u_{xx} + u_{yy} = 2+4=6\neq 0$, so $u$ is not harmonic.
        \item $u(x,y)=\sin(x)\cosh(y)$
        \begin{align*}
            u_x&=\cos(x)\cosh(y) & u_y&=\sin(x)\sinh(y) \\
            u_{xx}&=-\sin(x)\cosh(y) & u_{yy}&=\sin(x)\cosh(y) \\
        \end{align*}
        $\Delta u = u_{xx} + u_{yy} =-\sin(x)\cosh(y)+\sin(x)\cosh(y)=0$, so $u$ is harmonic.
    \end{enumerate}
    When a function $u(x,y)$ is not harmonic, then it can not be the real part of an analytic function. In case it is harmonic, then it can be the real part of an analytic function, but it is not necessary.\\
    Applying this for the previous functions:
    \begin{enumerate}[(i)]
        \item As $u(x,y)$ in this case was not harmonic, it is not the real part of an analytic function.
        \item As $u(x,y)$ in this case was harmonic, to determine if it is the real part of an analytic function we will need to search for that function explicitly. 
        \\Let us suppose $f(x,y)=u(x,y)+iv(x,y)$ is the analytic function we are looking for. As it is analytic, then it satisfies the Cauchy-Riemann equations (part a):\\
        \begin{equation*}
            u_x=\cos(x)\cosh(y)=v_y
        \end{equation*}
        \begin{equation*}
            u_y=\sin(x)\sinh(y)=-v_x
        \end{equation*}
        Taking the first equation and integrating:\\
        \begin{equation*}
            v(x,y)=\int_{0}^y\cos(x)\cosh(t)dt+h(x)=\cos(x)\sinh(y)+h(x)
        \end{equation*}
        Now differentiating it in function of $x$ and applying the second equation:\\
        \begin{equation*}
            v_x=-\sin(x)\sinh(y) + h'(x) = -\sin(x)\sinh(y)
        \end{equation*}
        Taking everything into consideration we obtain that $v(x,y)=\cos(x)\sinh(y)$ and $f(x,y)= \sin(x)\cosh(y)+i\cos(x)\sinh(y)$. So yes, $u(x,y)$ is the real part of an analytic function.
    \end{enumerate}
    \begin{tcolorbox}[title=Part c]
        Find a number $a\in\R$ such that
        \begin{align*}
            f(z)=f(x+iy)=\exp{axy}\cos(x^2-y^2)+i\exp{axy}\sin(x^2-y^2),
        \end{align*}
        is an entire function.
    \end{tcolorbox}
    Being an entire function means that $f(z)$ is analytic in $\C$. So we will suppose that $f(z)$ is differentiable in $\C$, and then the Cauchy-Riemann equations must hold.\\
    \begin{equation*}
        u(x,y)=\exp{axy}\cos(x^2-y^2), \quad v(x,y)=\exp{axy}\sin(x^2-y^2)
    \end{equation*}
    \begin{align*}
        u_x&=\exp{axy}(ay\cos(x^2-y^2)-2x\sin(x^2-y^2))\\
        u_y&=\exp{axy}(ax\cos(x^2-y^2)+2y\sin(x^2-y^2)) \\
        v_x&=\exp{axy}(ay\sin(x^2-y^2)+2x\cos(x^2-y^2))\\
        v_y&=\exp{axy}(ax\sin(x^2-y^2)-2y\cos(x^2-y^2)) \\
    \end{align*}
    Applying the equations:\\
    \begin{align*}
        u_x=v_y \Rightarrow a=-2\\
        u_y=-v_x \Rightarrow a=-2\\
    \end{align*}
    \end{document}
